\subsection{Matematické modelovanie zariadenia}
Najjednoduchším matematickým modelom, ktorý opisuje prietokový biochemický reaktor je tzv. Monod model. Tento model je veľmi obľúbený hlavne kvôli svojej jednoduchosti. Zakladá sa na dvoch predpokladoch: \text{1)} špecifická rýchlosť rastu buniek závisí od koncentrácie substrátu a \text{2)} tvorba biomasy je spojená so spotrebou substrátu. Formulácia rovníc, ktoré popisujú materiálovú bilanciu biomasy je nasledovná
\begin{table}[H]
	\centering
	\begin{tabular}{ccccc}
		akumulácia & & množstvo & & množstvo \\
		bunkovej & = & vzniknutých & -- & odobraných \\
		hmoty & & buniek & & buniek \\
	\end{tabular}
\end{table}
a pre materiálovú bilanciu substrátu platí:
\begin{table}[H]
	\centering
	\begin{tabular}{ccccccc}
		akumulácia & & množstvo & & množstvo & & množstvo\\
		substrátu & = & dodaného & -- & odobraného & -- & spotrebovaného .\\
		v systéme & & substrátu & & substrátu & & substrátu MO\\
	\end{tabular}
\end{table}
Ak uvažujeme, že objem reaktora sa nemení a prítok substrátu sa rovná odtoku suspenzie, potom môžme písať
\begin{align}
	&V\left(\der{x}{t}\right) = V\mu(s)x - Fx, \label{eq:tmp_monod_biomass} \\
	&V\left(\der{s}{t}\right) = Fs_{in} - Fs - V\frac{1}{Y_{x}}\mu(s)x. \label{eq:tmp_monod_subs}
\end{align}
Obe strany rovníc vydelíme objemom reaktora a označíme si pomer $ F/V = D $ ako rýchlosť riedenia, môžeme rovnice \eqref{eq:tmp_monod_biomass} a \eqref{eq:tmp_monod_subs} upraviť do nasledovného tvaru
\begin{align} 
	&\der{x}{t} = \left(\mu(s) - D\right)x, \text{kde}  \qquad \mu(s) = \mu_{m}\frac{s}{K_{M} + s}, \label{eq:monod_biomas}\\
	&\der{s}{t} = D\left(s_{in} - s\right) - \frac{1}{Y_{x}}\mu(s)x. \label{eq:monod_subs}
\end{align}

\subsubsection{Základné mechanické modely biochemického reaktora}
Rovnice \eqref{eq:monod_biomas} a \eqref{eq:monod_subs} tvoria najjednoduchší opis biochemického reaktora -- Monod model, a význam jednotlivých parametrov je uvedený v Tabuľke \ref{tab:monod_params}. Avšak, tento model má množstvo nedostatkov. Nedokáže vysvetliť jednotlivé fázy rastu, ktoré sú pozorované experimentálne a to: lag-fázu, smrť buniek na základe hladovania, tvorbu produktu atď. Tieto nedostatky boli doplnené u tzv. štrukturovaných modelov. Model, ktorý berie do úvahy aj tvorbu produktu, získame doplnením Monod modelu
\begin{align} 
&\der{x}{t} = \left(\mu(s) - D\right)x, \\
&\der{s}{t} = D\left(s_{in} - s\right) - \frac{1}{Y_{x}}\mu(s)x - \frac{1}{Y_{p}}\nu x, \\
&\der{p}{t} = \nu x - Dp, \label{eq:monod_product}
\end{align}
kde $p$ predstavuje koncentráciu produktu v \si{\gram\per\liter}, $Y_{p}$ je bezrozmerový koeficient výťažnosti produktu a $\nu$ predstavuje kinetický člen rýchlosti tvorby produktu v jednotkách času napr. \si{\per\hour}. Do rovnice \eqref{eq:monod_subs} sme doplnili časť, ktorá vraví, že časť substrátu sa spotrebuje na tvorbu produktu a rovnica \eqref{eq:monod_product} predstavuje obyčajnú materiálovú bilanciu produktu. 

\begin{table}
	\centering
	\caption{Parametre Monod modelu, ich symbol a rozmer.}
	\label{tab:monod_params}
	\begin{tabular}{lll}
		\hline
		\textbf{Parameter} & \textbf{Symbol} & \textbf{Rozmer} \\
		\hline
		Špecifická rýchlosť rastu & $\mu(s)$ & \si{\per\hour} \\
		Maximálna špecifická rýchlosť rastu & $\mu_{m}$ & \si{\per\hour} \\
		Michaelisova konštanta & $K_{M}$ & \si{\gram\per\liter} \\
		Výťažok (biomasa) & $Y_{x}$ & \\
		Objem reaktora & $V$ & \si{\liter} \\
		Prietok substrátu/suspenzie & $F$ & \si{\liter\per\hour} \\
		Koncentrácia biomasy & $x$ & \si{\gram\per\liter} \\
		Koncentrácia substrátu & $s$ & \si{\gram\per\liter} \\
		Koncentrácia čerstvého substrátu & $s_{in}$ & \si{\gram\per\liter} \\
		\hline
	\end{tabular}
\end{table}

Ak by sme chceli do modelu zakomponovať tendenciu úmrtia mikroorganizmov v dôsledku príliš vysokej koncentrácie substrátu (vplyv osmotického tlaku), treba upraviť špecifickú rýchlosť rastu $\mu(s)$ tak, že bude obsahovať inhibičný člen $ K_i $, ktorého rozmer je \si{\gram\per\liter}. Špecifická rýchlosť rastu potom nadobudne tvar
\begin{equation}
\mu(s) = \mu_{m}\frac{s}{K_{M} + s + \frac{s^2}{K_i}}
\end{equation}
a model, ktorého špecifická rýchlosť rastu má takéto vlastnosti, sa zvykne nazývať inhibičný alebo tiež Haldane model. 

\subsection{Analýza matematických modelov - Dynamika, stabilita}

\subsection{Optimalizačné metódy}
\subsection{Prístupy k odhadu parametrov}