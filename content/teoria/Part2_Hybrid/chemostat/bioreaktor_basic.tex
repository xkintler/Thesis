\chapter{Optimalizácia biochemického reaktora}
V tejto časti sa budeme venovať, ako už napovedá aj samotný názov, ekonomickej optimalizácii konkrétneho dynamického systému a to prietokového biochemického reaktora. Prečo? Je na to niekoľko dôvodov. V prvom rade sa biochemické reaktory považujú za dôležitú súčasť chemického priemyslu. Široká škála dôležitých zlúčenín ako farmaceutické produkty, rôzne polyméry alebo produkty potravinárskeho priemyslu sa vyrábajú pomocou určitého fermentačného média (rôzne baktérie, kvasinky, vláknité huby alebo enzýmy) za prísne stanovených podmienok v biochemickom reaktore~\cite{srinivasan:chemostat_opt:2003}. V druhom rade, biochemické reaktory vykazujú širokú škálu dynamického správania a ponúkajú veľa problémov s modelovaním v dôsledku prítomnosti živých organizmov (mikroorganizmov), ktorých rýchlosť rastu je opísaná komplexnými kinetickými výrazmi~\cite{psichogios:hybrid_process_model:1992}. A v neposlednom rade, optimalizácia prietokového biochemického reaktora je aj jedným z hlavných cieľov tejto práce.

\section{Biochemický reaktor - Základné informácie}
Biochemický reaktor sa dá vo všeobecnosti definovať ako nádoba, ktorá využíva aktivitu biologického katalyzátora na dosiahnutie požadovanej chemickej premeny~\cite{kaushik:bioreactors:2014}.
Biochemický reaktor všeobecne poskytuje biomechanické a biochemické prostredie, ktoré riadi prenos živín a kyslíka do buniek a produkty metabolizmu z buniek. Dal by sa tiež označiť ako zariadenie, určené na optimálny rast a metabolickú aktivitu organizmu, pôsobením biokatalyzátora, enzýmu alebo mikroorganizmov a buniek zvierat alebo rastlín. Surovinou môže byť organická alebo anorganická chemická zlúčenina alebo dokonca komplexný materiál. Produkt konverzie môže zahŕňať pekárske kvasinky, proteín, štartovacie kultúry alebo primárne metabolity (napr. aminokyseliny, organické kyseliny, vitamíny, polysacharidy, etanol atď.) a sekundárne metabolity (napr. antibiotiká). Biochemické reaktory sa môžu použiť na biokonverziu alebo biotransformáciu produktov (steroidná biotransformácia, L-sorbitol), enzýmov (amyláza, lipáza, celuláza), rekombinantných produktov (niektoré vakcíny, hormóny, ako je inzulín a rastové hormóny). Biochemický reaktor musí byť navrhnutý tak, aby vyhovoval konkrétnemu procesu~\cite{kaushik:bioreactors:2014}.

\subsection{Rozdelenie biochemických reaktorov}
Na základe spôsobu prevádzky môže byť biochemický reaktor klasifikovaný ako vsádzkový, kontinuálny a polovsádzkový.
 
Pri vsádzkovom spôsobe sa sterilné kultivačné médium naočkuje mikroorganizmami. Počas tohto reakčného obdobia sa s časom menia množstvá buniek, substrátu vrátane výživných solí, vitamínov a produktov. Fermentácia prebieha vopred stanovenú dobu a produkt sa zozbiera na konci.

V polovsádzkovom režime sa do reaktora postupne pridávajú živiny, ako prebiehajú biochemické reakcie, aby sa získali lepšie výťažky a vyššia selektivita spolu s reguláciou reakčnej teploty. Produkty sa zbierajú na konci výrobného cyklu ako pri vsádzkovom biochemickom reaktore. Charakteristickou črtou kontinuálneho biochemického reaktora je proces neustáleho dodávania substrátu. Prúd kvapaliny alebo suspenzie sa kontinuálne privádza a odstraňuje z reaktora. Na dosiahnutie rovnomerného zloženia a teploty je potrebné mechanické alebo hydraulické miešanie. Kultivačné médium, ktoré je buď sterilné alebo obsahuje mikroorganizmy, sa nepretržite dodáva do biochemického reaktora, aby sa udržal stabilný stav. Reakčné premenné a monitorovacie parametre zostávajú konzistentné a vytvárajú v reaktore časovo konštantný stav. Výsledkom je nepretržitá produktivita.

Tradičné vsádzkové miešacie tankové reaktory (STR) a kontinuálne miešané tankové reaktory (CSTR) sú široko prijímané v chemickom a biologickom priemysle kvôli ich jednoduchosti. Existujú aj iné biochemické reaktory, ktoré majú špeciálne konštrukčné a prevádzkové vlastnosti ako foto-bioreaktory, rotačné bubnové reaktory, hmlový, membránový biochemický reaktor, reaktory s náplňou a fluidnou vrstvou atď. Tieto boli navrhnuté tak, aby vyhovovali špecifickým procesom~\cite{kaushik:bioreactors:2014}.

\subsection{Parametre opisujúce biochemický reaktor}
Hlavné premenné, ktoré opisujú mikrobiálne procesy v prírode sú uvedené v Tabuľke \ref{tab:chemostat_dyn_param}.

\textbf{Množstvo mikroorganizmov} môže byť vyjadrené ako biomasa $x$ alebo počet buniek $N$ pri jednobunkových organizmoch (baktérie, kvasinky, spóry) na jednotku pôdy, množstva vody, objemu alebo obsahu plochy. Vláknité organizmy (huby, aktinomycéty) sú charakterizované dĺžkou mycélia $L$ a počtom hýf $n$. Treba zdôrazniť, že $n$ nie je totožné s $N$, pretože vetvenie hýf skôr pripomína delenie buniek pri jednobunkových organizmoch. Vzťah medzi $x$, $N$ a $L$ nie je jednoznačný pretože hmota jednotlivých buniek a šírka hýf sa líši v závislosti od organizmu a podmienok rastu. Všeobecne možno povedať, že pri nadbytku výživných zlúčenín sa formujú veľké bunky resp. široké hýfy, zatiaľ čo pri hladovaní sa tvoria skôr menšie bunky alebo užšie hýfy. Výber biomasy $x$ alebo počet buniek $N$ alebo dĺžku mycélia $L$ závisí na danom prípade. Biomasa $x$ má očividnú výhodu pri skúmaní cyklu uhlíka a živín, zatiaľ čo počet buniek $N$ sa preferuje pri skúmaní populácie napr. výskyt mutácií alebo prenos plazmidov~\cite{panikov:kinetics_MO_processes:2016}.

\begin{table}
	\centering
	\caption{Prehľad hlavných dynamických parametrov opisujúcich biochemický reaktor \cite{panikov:kinetics_MO_processes:2016}.}
	\label{tab:chemostat_dyn_param}
	\begin{tabular}{p{5cm} p{1.9cm} p{4cm}}
		\hline
		\textbf{Parameter} & \textbf{Symbol} & \textbf{Rozmer} \\ 
		\hline
		Hustota/koncentrácia biomasy & $x$ & \si{\micro\gram} bunkovej hmoty na \si{\gram} pôdy; \si{\gram} bunkovej hmoty \si{\per\square\meter} pôdy; \si{\micro\gram} bunkovej hmoty na \si{\milli\liter} vody\\
		Počet buniek & $N$ & $10^{6}$ buniek na \si{\gram} pôdy; $10^{6}$ buniek na \si{\milli\liter} vody\\
		Dĺžka mycélia & $ L $ & \si{\meter} na \si{\gram} pôdy; \si{\meter} na \si{\milli\liter} vody\\
		Počet hýf & $n$ & $10^{6}$ na \si{\gram} pôdy; $10^{6}$ na \si{\milli\liter} vody\\
		Koncentrácia limitujúceho substrátu & $s$ & \si{\milli\gram} na \si{\gram} pôdy; \si{\gram\per\square\meter} pôdy; \si{\gram\per\liter}vody\\
		Koncentrácia produktu & $p$ & \si{\milli\gram} na \si{\gram} pôdy; \si{\gram\per\square\meter} pôdy; \si{\gram\per\liter}vody\\
		\hline	
	\end{tabular}
\end{table}

\textbf{Koncentrácia limitujúceho substrátu} vo vode alebo v pôde, $(s)$, predstavuje množstvo esenciálnej živiny využívanej mikroorganizmami na rast a rozmnožovanie. Bežne nevieme posúdiť všetky potenciálne dostupné živiny a zamerať sa iba na jednu alebo zopár individuálnych zlúčenín alebo triedu molekúl, ktorá reprezentuje limitujúcu zlúčeninu, pretože chemoorganotrofné mikroorganizmy čerpajú energiu z organických zlúčenín, zatiaľ čo fotosyntetizujúce mikroorganizmy vyžadujú prísun svetla a zdroj fosforu, dusíka a železa~\cite{panikov:kinetics_MO_processes:2016}.

\textbf{Množstvo produktov} $(p)$. Sem patria všetky medziprodukty a konečné produkty metabolizmu mikroorganizmov, ktoré vznikajú počas rastu. Typickými medziproduktmi sú organické kyseliny vznikajúce počas glykolýzy. Jediný konečný produkt aeróbnej mikrobiálnej dekompozície je oxid uhličitý, avšak pri anaeróbnych podmienkach vznikajú rôzne organické kyseliny, alkoholy, ketóny atď~\cite{panikov:kinetics_MO_processes:2016}.
