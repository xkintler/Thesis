\part{Hybridné modelovanie a jeho využitie v praxi}
\chapter{Hybridné modelovanie}
Úlohou modelovania procesov je získať matematický predpis na základe znalostí, ktoré o tomto procese máme \cite{hangos:process_modelling:2001}. V závislosti od prístupu k modelovaniu, môžeme získané modely rozdeliť do viacerých skupín. Prvou skupinou sú tak zvané \aps{mechanické} modely. Tieto sú odvodené z fyzikálnych zákonov, ktoré predstavujú rôzne zákony zachovania -- bilancie, hmoty alebo energie, zákony kinetiky, termodynamiky, prestupu látky atď \cite{bangi:chem_engineer:2020}. Takéto modely sú transparentné a ľahko pochopiteľné, pretože majú za sebou skutočnú fyzikálnu podstatu, ktorá platí pre široké spektrum operačných podmienok. Nevýhodou býva, že často sú veľmi zložité a samotné modelovanie je náročné na čas. Druhú skupinu tvoria dátové modely, ktorých problematiku sme rozobrali v predchádzajúcich kapitolách. Spomenieme, že majú viacero výhod -- sú jednoduché na získanie, čím ušetríme kopec času s modelovaním, často majú jednoduchšiu štruktúru, sú flexibilnejšie atď. Nevýhodou však je, že ich štruktúra nám neprezradí nič o samotnej povahe procesu. Hybridné modely tvoria tretiu skupinu a  sú kombináciou mechanických a dátových modelov, pričom využívajú výhody z obidvoch skupín, čím našli široké uplatnenie v rôznych oblastiach -- bioinžinierstvo \cite{srivastava:hybrid_biomolecules:2020}, strojníctvo \cite{liu:hybrid_vehicle:2020}, životné prostredie \cite{liu:hybrid_waste_water:2019}, energetika \cite{qian:hybrid_energy:2019} atď.

\section{Hybridné modely}
Základy hybridného modelovania položili Psichogios a Ungar v práci \aps{\textit{A hybrid neural network-first principles approach to process modeling}} z roku 1992 \cite{psichogios:hybrid_process_model:1992}. Ich cieľom bolo vytvoriť hybridný model založený na neurónovej sieti a mechanickom modely vsádzkového biochemického reaktora. Vo výsledku sa im podarilo zlepšiť presnosť predikcie v porovnaní so samotným mechanickým modelom, dosiahnuť lepšiu interpoláciu a extrapoláciu na rozdiel od samotnej neurónovej sieti a výrazne sa uľahčila analýza a interpretácia dát.

Existuje veľa rôznych kombinácii mechanických a dátových modelov, ktoré vedú k ešte väčšiemu množstvu hybridných modelov, ale vo všeobecnosti by sme mohli takýto model sformulovať nasledovne
\begin{equation}
	\begin{split}
		\der{x}{t} &= f(t,x,u,\theta), \\
		\theta &= g(x,u),
	\end{split}
\end{equation}
kde $ x $ predstavuje stavy, $ u $ vstupy, $ \theta $ parametre procesu. Celá dynamika systému je definovaná funkciou $ f $, ktorá predstavuje základný mechanický model a je závislá na parametroch systému $ \theta $, ale tie zasa závisia od stavov $ x $ a vstupov $ u $, ktoré vieme získať z dátovej časti. Takáto štruktúra hybridného modelu je zobrazená aj na Obr. \ref{fig:hybrid_model_general}.

\begin{figure}
	\centering
	\includegraphics[width=0.5\linewidth]{images/hybrid_model}
	\caption{Príklad všeobecného hybridného modelu.}
	\label{fig:hybrid_model_general}
\end{figure}