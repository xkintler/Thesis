\chapter{Záver}
Mechanistické modelovanie môže byť veľmi komplikovaná záležitosť, najmä ak sa snažíme matematicky opísať komplexné zariadenia. Na druhej strane, výsledné modely sú transparentné a ľahko pochopiteľné, pretože majú za sebou skutočnú fyzikálnu podstatu. Tento prístup k modelovaniu má však viacero nevýhod akými sú časová aj finančná náročnosť a skutočnosť, že po čase nemusí matematický model zodpovedať reálnemu zariadeniu kvôli zmenám v prevádzke zariadenia. 

Dátové modelovanie dokáže obísť všetky spomenuté problémy, plus dátové modely majú výrazne jednoduchšiu štruktúru a sú flexibilnejšie. Ale ich štruktúra nám nič neprezradí o samotnej povahe procesu. S dátovým modelovaním sa spájajú ešte ďalšie problémy ako voľba správnej štruktúry dátového modelu a odhad jeho parametrov. Väčšina konvenčných metód nerieši problematiku voľby vhodnej štruktúry modelu, pričom táto problematika je rovnako dôležitá a výrazne komplikovanejšia ako odhad parametrov modelu. Metóda garantovaného odhadu nám ponúka informácie o minimálnom ráde (štruktúre) modelu a v spojení s Pareto frontom sme dokázali posúdiť aj kvalitu modelov vyšších rádov. Vhodný rád modelu sme potom zvolili na základe kompromisu medzi presnosťou odhadu modelu a jeho maximálnym rozptylom odhadu. V ďalšej časti identifikácie pomocou GOP sme získali garantovanú oblasť všetkých možných riešení, v rámci stanovenej chyby modelu. Táto oblasť, ktorá je určená minimálnou a maximálnou realizáciou modelu nám zaručuje, že skutočné riešenie leží vo vnútri.

Hybridné modely predstavujú kombináciu mechnistických a dátových modelov, pričom využívajú výhody z obidvoch skupín --- v porovnaní so samotným mechanistickým modelom vykazujú presnejšie predikčné vlastnosti a na rozdiel od samotného dátového modelu dosahujú lepšiu interpoláciu aj extrapoláciu a interpretácia a analýza dát sú výrazne ľahšie. Z týchto dôvodov si našli široké uplatnenie v oblasti automatizácie. Optimalizácia v reálnom čase zahŕňa metódy, ktoré sa dokážu vysporiadať s problematikou odlišnosti správania zariadenia a jeho mechanistického modelu. My sme spomenuli dvojkrokovú optimalizáciu, schému úpravy modifikátora a hybridné modely. V tejto práci sme sa rozhodli demonštrovať funkčnosť jednotlivých metód pri optimalizácii prevádzky prietokového biochemického reaktora, pretože ponúkajú veľa problémov s modelovaním v dôsledku prítomnosti živých organizmov. Výsledky práce nás priviedli k záveru, že dvojkroková optimalizácia nie je vhodný nástroj na optimalizáciu prietokového biochemického reaktora, pretože vo väčšine prípadov uviedla zariadenie do stavu vymytia. Schéma úpravy modifikátora a hybridné modelovanie si viedli veľmi podobne. V priemere nás dostali do približne rovnakého okolia od optimálneho ustáleného stavu. Ale veľký rozdiel bol v rýchlosti konvergencie. Zatiaľ čo metóde schéme úpravy modifikátora to v priemere trvalo celých 20 iterácii, čo v prípade nami zadefinovaného biochemického reaktora predstavovalo 1000 hodín, hybridné modely to zvládli za 5 iterácií, teda 250 hodín. Výrazný rozdiel oboch metód bol aj v citlivosti na šum merania, kde hybridné modeli jasne napredovali.

Touto diplomovou prácou sme ukázali, že dokážeme aplikovať metódu garantovaného odhadu parametrov pri hybridnom modelovaní, a že takýto prístup vieme následne použiť pri optimalizácii prevádzky zariadenia. Víziou budúcej práce by mohla byť aplikácia takto zhotovených hybridných modelov pri riadení dynamických systémov, kde by mohli nájsť väčšie uplatnenie.  

