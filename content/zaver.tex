\chapter{Záver}
Odlišnosť v správaní skutočného biochemického reaktora a jeho modelu predstavuje závažný problém, hlavne v situácii, keď sa snažíme zabezpečiť efektívnu prevádzku zariadenia. Veľa vedeckých publikácii sa venuje práve tejto tématike, ale riešenia, ktoré ponúkajú, ako napr. dvojkroková optimalizácia alebo schéma úpravy modifikátora, sú často veľmi komplikované na realizáciu alebo vedú k nepresným výsledkom. Táto práca sa zaoberala odlišným prístupom k problematike optimalizácie prietokového biochemického reaktora, pomocou nepresného mechanistického modelu, ktorý bol založený na hybridnom modelovaní s využitím garantovaného odhadu parametrov.

Výsledky experimentov nás priviedli k záveru, že hybridné modelovanie môže byť použité na optimalizáciu prevádzky biochemického reaktora. Z dát, ktoré sme dokázali získať zo zariadenia, sme odvodili dva hybridné modely, jeden substrátový a druhý biomasový. Oba viedli k rovnakým výsledkom, ak sme upravili veľkosť chyby merania koncentrácie biomasy tak, aby bola porovnateľná s chybou merania koncentrácie substrátu, vzhľadom na generované skokové zmeny. Uviedli sme niekoľko prístupov v rámci hybridného modelovania, ktoré sa dajú aplikovať na spomínanú problematiku. Iteračná metóda dokázala skonvergovať v priebehu pár krokov, ale dostala sa iba do okolia optima. Ak sme dátovú časť hybridného modelu natrénovali na dopredu známych dátach z viacerých skokových zmien, tak sme získali presné optimum zariadenia, ktoré sme dosiahli iba vďaka nepresnosti odhadu. Najdôležitejším výsledkom v rámci hybridného modelovania je, že aj keby hybridný model predikoval presný rozdiel v koncentrácii zariadenia a nominálneho modelu v optimálnom režime, nikdy by sme nedosiahli skutočné optimum zariadenia, ale dostali by sme sa do veľmi blízkeho okolia.

Pri porovnaní jednotlivých metód pre niekoľko rôznych realizácií šumu merania sme zistili, že vo väčšine prípadov si s problematikou optimalizácie prevádzky prietokového biochemického reaktora najlepšie poradila metóda s použitím hybridného modelovania, i keď od začiatku sme vedeli, že nás dostane iba do blízkeho okolia optima. V dvadsiatej iterácii sa substrátový hybridný model aj schéma úpravy modifikátora dostali do približne rovnakého ustáleného stavu. Rozdiel bol v tom, že zatiaľ čo metóde schéme úpravy modifikátora to v priemere trvalo celých 20 iterácii, čo v prípade nami zadefinovaného biochemického reaktora predstavuje 1000 hodín, hybridné modely to zvládli za 5 iterácií, teda 250 hodín. Dvojkroková optimalizácia už v druhej iterácii uviedla zariadenie do optimálneho ustáleného stavu, ale v ďalších krokoch divergovala od optima do stavu vymytia. V porovnaní s dvojkrokovou optimalizáciou a schémou úpravy modifikátora bola menšia citlivosť na šum merania veľkou výhodou pre hybridné modely.

Ukázali sme, že korekcia ustálených stavov nominálneho modelu pomocou hybridných modelov, nie je v spojní s iteračným prístupom najvhodnejším adeptom na optimalizáciu prevádzky. Tento prístup by sme mohli vylepšiť zmenou štruktúry hybridného modelu za sériovo-paralelnú, čím by sme dokázali obísť problém s nelinearitou systému, čo by viedlo k lepším predikčným vlastnostiam. Výraznejšie zlepšenie optimalizačného procesu by sme mohli dosiahnuť zmenou prístupu k optimalizácii v rámci hybridného modelovania, napr. podobne ako to rieši schéma úpravy modifikátora, t.j. lineárnou korekciou gradientu účelovej funkcie. Tieto všetky hypotézy by však bolo nutné overiť, čo by mohlo byť námetom na ďalšiu vedeckú činnosť. Predikčné vlastnosti hybridných modelov, ktoré sú omnoho lepšie v porovnaní so samotným nominálnym modelom, by sme vedeli využiť aj v iných oblastiach automatizácie ako napr. riadenie. 