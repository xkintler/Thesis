Odlišné správanie skutočného zariadenia a jeho matematického opisu je v oblasti automatizácie pomerne známa záležitosť, ktorá predstavuje závažný problém, najmä v situácii, keď sa snažíme zabezpečiť optimálnu prevádzku zariadenia. Veľa vedeckých publikácií sa venuje práve tejto problematike, ale riešenia, ktoré ponúkajú sú často veľmi komplikované na realizáciu alebo vedú k neistým výsledkom. Táto práca prináša nový prístup k dynamickej optimalizácii zariadenia, ktorý je založený na hybridnom modelovaní s použitím metódy garantovaného odhadu parametrov. Funkčnosť tejto metódy sme sa rozhodli demonštrovať na zariadení, ktoré predstavuje prietokový biochemický reaktor, pretože ponúka veľa problémov s modelovaním v dôsledku prítomnosti živých organizmov. Výsledky experimentov ukázali, že hybridné modely môžu byť použité na optimalizáciu prevádzky zariadenia, ale iteračným prístupom nás dokážu dostať iba do blízkeho okolia optima zariadenia a to v závislosti od hladiny šumu. Ale na rozdiel od iných metód je konvergencia hybridných modelov výrazne rýchlejšia a menej citlivá na šum merania, čo je obrovskou výhodou pre procesy s veľkými časovými konštantami.

% RP: Ja si myslim, ze to je "mechanisticky" nie "mechanicky", ale mozem sa mylit
