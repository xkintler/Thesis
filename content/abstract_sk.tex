% RP: Abstrakt skor kratsi ako dlhsi. Ziadne velke detaily. Ako zvykol hovorit doc. Dvoran: "Pohlad z lietadla ked je zamracene". V podstate stacia 4 vety (1. studujeme toto, 2. robime to takto, 3. demonstrujeme to takto, 4. vysledky ukazuju toto) ale moze sa to aj trocha rozviest.

Odlišné správanie skutočného zariadenia a jeho matematického opisu je v oblasti automatizácie pomerne známa záležitosť, ktorá predstavuje závažný problém, najmä v situácii, keď sa snažíme zabezpečiť optimálnu prevádzku zariadenia. Veľa vedeckých publikácií sa venuje práve tejto problematike, ale riešenia, ktoré ponúkajú sú často veľmi komplikované na realizáciu alebo vedú k neistým výsledkom.
% RP: Nasledujuca veta je kostrbata...povie ze ideme pouzivat nepresny model ale to znie zvlastne lebo clovek by cakal, ze ten model dajak opravime...toto je vlastne zhrnutie hlavneho prinosu a musit byt "extra sharp"
Táto práca prináša nový prístup k optimalizácii zariadenia pomocou nepresného
% RP: Ja si myslim, ze to je "mechanisticky" nie "mechanicky", ale mozem sa mylit
mechanického modelu, ktorý je založený na hybridnom modelovaní s použitím metódy garantovaného odhadu parametrov. Funkčnosť tejto metódy sme sa rozhodli demonštrovať na zariadení, ktoré predstavuje prietokový biochemický reaktor, pretože ponúka veľa problémov s modelovaním v dôsledku prítomnosti živých organizmov.
% RP: Nasledujuca veta je uz navyse, nehodi sa do abstraktu. Ked uz, tak ju zaradit pred bioreaktor a urobit ju nezavislu na nejakych "bioreaktorovych detailoch".
Základnou myšlienkou je doplniť vzniknuté nezhody medzi skutočným zariadením (Monod model) a nepresným mechanickým modelom (Haldane model) pomocou dátových modelov, ktoré boli identifikované práve na týchto údajoch rozdielov.
Výsledky experimentov ukázali, že hybridné modely môžu byť použité na optimalizáciu prevádzky zariadenia, ale iteračným prístupom
% RP: Neviem ci "nie su schopne" nie je prilis negativny pohlad na vec. Poznate nejaku lepsiu metodu? Podla mna sa ta metoda sprava dobre alebo zle v zavisloti od hladiny sumu a v zavisloti od dostupnosti meranych premennych
nie sú schopné zabezpečiť konvergenciu ku skutočnému optimálnemu ustálenému stavu zariadenia, iba k jeho blízkemu okoliu. Na rozdiel od schémy
% RP: Nie som si isty tym prekladom "úprava modifikátora", skor "uprava modifikatorov". Je tu ale este problem, ze v abstrakte nikto nevie co RTO alebo MA a malo by sa tu teda skor hovorit o porovnani s inymi metodami vseobecne (robili ste aj two-step approach a ten sa tu vobec nezjavuje...nehovorim)
úpravy modifikátora, je konvergencia hybridných modelov výrazne rýchlejšia a menej citlivá na šum merania, čo je obrovskou výhodou pre procesy s veľkými časovými konštantami.
