The different behavior of the real plant and its mathematical description is a relatively well-known issue in the field of automation, which is a serious problem, especially in a situation where we try to ensure optimal operation of the plant. Many scientific publications address this issue, but the solutions they offer are often very complicated to implement or lead to uncertain results. This work brings a new approach to the model--plant mismatch optimization, which is based on hybrid modeling using the method of guaranteed parameter estimation. We decided to demonstrate the functionality of this method on a plant that is represented by a chemostat, because it offers many problems with modeling due to the presence of living organisms. The basic idea is to supplement the discrepancies between the real plant (Monod model) and the inaccurate mechanical model (Haldane model) using data models, which were identified on the data described by these differences.

The results of the experiments showed that hybrid models can be used to optimize the operation of the plant, but  with an iterative approach they are not able to ensure convergence to the real optimal steady state of the plant, only to its immediate surroundings. Unlike the modifier adaptation scheme, the convergence of hybrid models is significantly faster and less sensitive to measurement noise, which is a great advantage for processes with large time constants.