\chapter{Resumé}
V oblasti automatizácie sa často stretávame s problematikou nezhody matematického opisu so skutočným zariadením. Tieto nezhody sú väčšinou spôsobené rôznymi aproximáciami, ktorými sa snažíme v jednoduchosti, matematicky opísať správanie komplexného systému. Podobne je to aj v prípade biochemických reaktorov. Vlastnosti biochemických reaktorov sú výlučne stanovené správaním živých organizmov, ktoré ponúkajú široké spektrum dynamického správania. Ich správanie sa líši nielen v závislosti od typu mikroorganizmu, ale aj v rámci jedného druhu, čo vedie k problémom s modelovaním takýchto procesov a v konečnom dôsledku môžeme skončiť s matematickým opisom, ktorý je rôzny od skutočnosti. 

Odlišnosť v správaní skutočného biochemického reaktora a jeho modelu predstavuje závažný problém, hlavne v situácii, keď sa snažíme zabezpečiť efektívnu prevádzku zariadenia. Veľa vedeckých publikácii sa venuje práve tejto tématike, ale riešenia, ktoré ponúkajú, ako napr. dvojkroková optimalizácia alebo úprava účelovej funkcie použitím modifikátora, sú často veľmi komplikované na realizáciu alebo vedú k nepresným výsledkom. Z tohto dôvodu sme sa rozhodli, že navrhneme odlišný prístup k problematike optimalizácie prietokového biochemického reaktora, pomocou nepresného mechanického modelu, ktorý je založený na hybridnom modelovaní s využitím garantovaného odhadu parametrov.

Väčšina metód nerieši problematiku voľby rádu modelu, ktorá je esenciálnou úlohou pri identifikácii dátových modelov a výrazne komplikovanejšou ako odhad parametrov modelu. Garantovaný odhad parametrov nám ponúka informácie o minimálnom ráde (štruktúre) modelu a v spojení s Pareto frontom sme dokázali posúdiť aj kvalitu modelov vyšších rádov. Vhodný rád modelu sme potom zvolili na základe kompromisu medzi presnosťou odhadu modelu a jeho maximálnym rozptylom odhadu. V ďalšej časti identifikácie pomocou GOP sme získali garantovanú oblasť všetkých možných riešení, v rámci stanovenej chyby modelu. Táto oblasť, ktorá je určená minimálnou a maximálnou realizáciou modelu nám zaručuje, že skutočné riešenie leží v vo vnútri.  

Ukázali sme, že takto dokážeme nájsť vhodné dátové modely, či už FIR alebo ARX, na opis údajov, ktoré vyjadrujú rozdiel koncentrácie biomasy alebo substrátu medzi skutočným zariadením (Monod model) a nominálnym modelom (Haldane model). Dynamika týchto modelov bola odlišná, ale v predikcií ustálených stavov boli FIR aj ARX model rovnako dobré. Preto sme sa rozhodli, že pri konštrukcii hybridných modelov sa zameriame iba na FIR modely.

Výsledky experimentov hybridného modelovania nás priviedli k záveru, že hybridné modelovanie môže byť použité na optimalizáciu prevádzky biochemického reaktora. Z dát, ktoré dokážeme získať zo zariadenia, sme odvodili dva hybridné modely, jeden substrátový a druhý biomasový. Oba viedli k rovnakým výsledkom, ak sme upravili vplyv chyby merania koncentrácie biomasy tak, aby bola porovnateľná s chybou merania koncentrácie substrátu, vzhľadom na generované skokové zmeny. Uviedli sme niekoľko prístupov v rámci hybridného modelovania, ktoré sa dajú aplikovať na spomínanú problematiku. Iteračná metóda dokázala skonvergovať v priebehu pár krokov, ale dostala sa iba do okolia optima. Ak sme dátovú časť hybridného modelu natrénovali na dopredu známych dátach, tak sme sa dostali do presného optimálneho stavu zariadenia. Ale ak by sme pridali dáta zo skokových zmien pri väčších hodnotách rýchlosti riedenia, tak s veľkou pravdepodobnosťou by sme optimálny stav presiahli. Nevýhodou tohto prístupu je aj skutočnosť, že ak zapojíme vopred natrénovaný model do iteračnej optimalizácie, tak po niekoľkých iteráciách sa ustálený stav zariadenia vzdiali od optimálneho. Je to spôsobené tým, že po určitom čase už iteračný prístup nedokáže vygenerovať informačne výdatné dáta na identifikáciu dátového modelu. Najdôležitejším výsledkom v rámci hybridného modelovania je, že aj keby hybridný model predikoval presný rozdiel v koncentrácii zariadenia a nominálneho modelu v optimálnom režime, nikdy by sme nedosiahli skutočné optimum zariadenia, ale dostali by sme sa do veľmi blízkeho okolia.

Dvojkroková optimalizácia dokázala v niektorých špeciálnych prípadoch (najmä na začiatku optimalizačného procesu, keď sme mali k dispozícii málo dát) skonvergovať k optimu zariadenia, iba pomocou odhadu kinetických členov nominálneho modelu, pretože zväčšovaním hodnoty inhibičného koeficientu, vieme zmenšiť rozdiely v správaní Monod a Haldane modelu.
V ďalších iteráciách, kedy narastal počet dát, teda sa zvyšovala presnosť odhadu, sa riešenia od optima vzdialili. Problematická situácia nastala najmä vtedy, ak koeficient inhibície $ K_I $ nadobudol hodnoty rádovo $ 10^{10}\si{\gram\per\liter} $. To viedlo k takému priebehu účelovej funkcie nominálneho modelu resp. iba jej časti, ktorá spôsobila problémy pri samotnom riešení optimalizačnej úlohy a výsledkom boli rýchlosti riedenia, ktoré dostali prietokový biochemický reaktor do stavu vymytia.

V teoretickej rovine, je schéma úpravy modifikátora jedinou z použitých metód, ktorá dokáže uviesť zariadenie do jeho optimálneho režimu v niekoľkých krokoch, aj napriek nepresnému matematickému opisu zariadenia. Rýchlosť konvergencie závisí od veľkosti váhového koeficientu $ c $, kde menšie hodnoty znamenajú rýchlejší priebeh konvergencie a naopak väčšie hodnoty pomalší. Situácia sa zmenila, ak sme začali uvažovať vplyv šumu merania. Ako sme ukázali, tak chyba merania mohla v niektorých situáciách pomôcť rýchlejšej konvergencii, ale ak bol šum merania výraznejší ako skoková zmena, vznikali oscilácie a nedokázali sme zabezpečiť konvergenciu k optimu, hlavne v iteráciách, kedy skutočný gradient účelovej funkcie bol rovný nule. Spoľahlivosť tejto metódy záleží na nastavení jej parametrov, čo vôbec nie je jednoduchá úloha, najmä v prípade biochemického reaktora, kde fluktuácia koncentrácie biomasy je veľmi výrazná.

Pri porovnaní jednotlivých metód pre niekoľko rôznych realizácií šumu merania sme zistili, že vo väčšine prípadov si problematikou optimalizácie prevádzky prietokového biochemického reaktora na základe nepresného mechanického modelu najlepšie poradila metóda s použitím hybridného modelovania, i keď od začiatku sme vedeli, že nás dostane iba do blízkeho okolia optima. V dvadsiatej iterácii sa hybridný model aj schéma úpravy modifikátora dostali do približne rovnakého ustáleného stavu. Rozdiel bol v tom, že zatiaľ čo metóde schéme úpravy modifikátora to trvalo celých 20 iterácii, čo v prípade biochemického reaktora predstavuje 1000 hodín, hybridné modely to zvládli za 5 iterácií, teda 250 hodín. Dvojkroková optimalizácia už v druhej iterácii uviedla zariadenie do optimálneho ustáleného stavu, ale v ďalších krokoch divergovala od optima do stavu vymytia.

Iteračný proces optimalizácie pomocou hybridných modelov by sa dal ešte vylepšiť. My sme ukázali, že korekcia ustálených stavov nominálneho modelu, nie je v spojní s iteračným prístupom najvhodnejším adeptom na optimalizáciu prevádzky. Ak by sme pristupovali k optimalizácii, v rámci hybridného modelovania, podobne ako schéma úpravy modifikátora, t.j. zmenou gradientu účelovej funkcie, je tu možnosť, že by sme takto dokázali dosiahnuť presné optimum. Podobne by sme mohli použiť hybridné modely na odhad ustálených stavov zariadenia, pomocou ktorých by schéma úpravy modifikátora vedela omnoho presnejšie odhadnúť gradient účelovej funkcie. Tieto hypotézy by však bolo potrebné overiť ďalším výskumom.

Hybridné modelovanie má niekoľko výhod oproti schéme úpravy modifikátora. Tým, že hybridné modely dokážu predikovať údaje zariadenia, vedeli by sme ich použiť pri riadení procesov, pričom riadenie by bolo omnoho kvalitnejšie ako s použitím samotného nominálneho modelu, čo sme aj demonštrovali na výsledkoch predikčných vlastností modelov. Šum merania nezohráva až takú významnú úlohu pri optimalizácii zariadenia. Dôležitú rolu pri tejto vlastnosti zohráva identifikácia dátových častí hybridných modelov pomocou metódy garantovaného odhadu parametrov. 

Kvalita hybridných modelov je stanovená kvalitou ich dátových častí. Ako sme ukázali, tak lineárne dátové modely ako FIR alebo ARX sa nedokážu efektívne vysporiadať s nelinearitou systému. Z tohto dôvodu vedú aj hybridné modely k nepresnej predikcii. S týmto problémom by sme sa pravdepodobne vedeli vysporiadať zmenou štruktúry hybridného modelu za sériovo--paralelnú, ale toto tvrdenie by bolo nutné overiť, čo môže byť námetom na ďalšiu prácu.